\chapter{总结和展望}
\section{论文工作总结}
针对云平台集成SDN的问题,本文创新性地提出了基于SDN的云平台多租户虚拟网络定制化机制,为租户提供了自我可控的SDN网络,实现了对网络的灵活自定义操作。完成的主要工作包括:

实现跨物理服务器的数据中心网络的集中控制,物理网络的集中控制,可以有效的减小网络配置的繁琐性,为数据中心规模的伸缩性提供了便利。通过虚拟化技术,为租户提供了相互隔离的vSDN网络,实现了数据中心侧的租户隔离,保证了数据传输的安全性。通过跨租户数据传输的测试,验证了租户之间的隔离性。

创建的vSDN网络由租户自由的控制器进行集中管控,现阶段,本文为云平台提供了Ryu控制器镜像,与此同时,为该控制器开发了北向应用,用于进行链路带宽和时延的测量,以及定制化流表的下发。租户可以根据当前链路的带宽、时延情况进行数据传输链路的选取,通过流表的下发,实现数据传输路径的定制化。当然,租户亦可以登录运行Ryu控制器的虚拟机,开发自己的北向应用,实现对数据中心虚拟网路的自定操作。通过对链路带宽、时延的测量,以及定制化流表的下发,验证了租户对数据中心虚拟网络的定制化。

为方便租户的可视化操作,本文提供了GUI模块,租户可以在web界面实现虚拟网络的创建,链路带宽、时延的查询,以及选路功能。实现租户对数据中心网络的灵活控制。

\section{存在的问题及展望}
本论文针对云数据中心网络,提出了基于SDN的云平台多租户虚拟网络定制化的机制,在一定程度上为租户提供了一种数据中心网络定制化的方案。但是还存在一些需要克服的缺点和需要改进的地方,因此在今后本研究工作将进一步关注以下几个问题:

\begin{enumerate}[1)]
\item 文中基于SDN的云数据中心网路,暂时仅支持二层转发,跨三层的数据传输,只能通过云平台传统模式实现,未来的研究需要实现跨三层的SDN数据传输。从而可以实现三层路由的链路定制化。
\item 本文对链路的选取,仅实现了手动选取的模式,未来需要实现链路的自动化选取,后台需要根据当前链路的带宽、时延,选取最优的数据传输路径。
\item 虽然系统实现了基本功能,但要想商业产品化,还有很多工作要做,如何高效稳定地实现租户云数据中心网络的定制化,也是进一步研究的问题之一。
\end{enumerate}