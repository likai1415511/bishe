%%
%% This is file `example/ackgmt.tex',
%% generated with the docstrip utility.
%%
%% The original source files were:
%%
%% install/buptgraduatethesis.dtx  (with options: `ackgmt')
%% 
%% This file is a part of the example of BUPTGraduateThesis.
%% 

\begin{acknowledgement}
  %% 感谢所有你应该感谢的人
从2013年9月份开始,我在北京邮电大学网络技术研究院网络管理中心的研究生生活便拉开了序幕。时光如指间流沙,两年半的研究生生涯转眼间就要结束,这段充实有意义的时光让我终身难忘。在这两年半的时间里,我从一个稚嫩的大学生一步一步成长为逻辑严谨、成熟稳重的硕士研究生,即将精神饱满地进入社会的摇篮。在学习和生活中,我不仅学到了很多计算机通信相关的专业基础知识,学会了如何在科学研究领域做学术研究,如何提升自己的理论水平和实践开发能力,更懂得了许多做人做事的道理。经历过研究生生活的酸甜苦辣,在此,我想对身边的每一个老师、同学、朋友和家人表达深深的感谢,谢谢你们对我给予的帮助和鼓励。

衷心感谢我的导师——廖建新老师。在我攻读硕士期间,感谢熊老师在学习、科研和生活中给予我的帮助、鼓励与指导。熊老师认真的工作精神、高效的办事效率、严谨的治学态度和渊博的知识不断地影响到我。读研期间,在任何时候您都能给予我及时的回复,并且每次都能让我受益匪浅。同时,尽管您业务繁忙,但您仍然对我的科研进展和学习生活情况关怀备至,令我十分感动。在今后的工作中,我会以您为榜样严格要求自己,积极向上。

衷心感谢负责我实验室日常工作的指导老师——王敬宇。王老师主要负责我科研方面的指导工作,从本科生毕业设计开始到研究生各阶段,王老师一点点开启了我的学术科研之路。不论是论文和专利撰写,还是标准制定、项目调研等一系列日常科研工作,王老师都给予我悉心指导。王老师具有踏实的工作态度和亲切的处事待人方式,是我们大家学习的榜样。王老师积极组织科研小组定期进行会议讨论,检查大家的科研进度,促进沟通和交流,使得我们获得了广阔和多层次的研究视角。浓郁的学术氛围极大地加快了大家的科研工作进度,使我们受益匪浅。王老师指导我参与了多项国家科研项目的研究,一直信任我、鼓励我,在我迷茫的时候指引我正确的研究方向。王老师就像一位辛勤的园丁,又像一位贴心的母亲,从您的身上我学到了温柔的力量。

衷心感谢负责我实验室日常工作的指导老师——戚琦老师。戚老师主要负责我科研方面的指导工作,从本科生毕业设计开始到研究生各阶段,戚老师一点点开启了我的学术科研之路。不论是论文和专利撰写,还是标准制定、项目调研等一系列日常科研工作,戚老师都给予我悉心指导。戚老师具有踏实的工作态度和亲切的处事待人方式,是我们大家学习的榜样。戚老师积极组织科研小组定期进行会议讨论,检查大家的科研进度,促进沟通和交流,使得我们获得了广阔和多层次的研究视角。浓郁的学术氛围极大地加快了大家的科研工作进度,使我们受益匪浅。戚老师指导我参与了多项国家科研项目的研究,一直信任我、鼓励我,在我迷茫的时候指引我正确的研究方向。王老师就像一位辛勤的园丁,又像一位贴心的母亲,从您的身上我学到了温柔的力量。

感谢实验室的同窗们。感谢龚军、孙海峰、王金柱、薛瑞等师兄师姐,在本科毕设和研究生前两年期间,你们帮助我走进网络管理中心的生活,让我感受到轻松愉快的学习氛围,走进科研的殿堂。感谢同组的李涛、何昱泽、陆中豪、陈良章,还有庄子睿、包剑楠、李钰剑、庞旭东、武莹等师弟师妹,有幸能和你们一起讨论学术问题,一起学习和玩耍,共同进步和成长。感谢李呈同学,虽不在同一实验室,他的博文成为我学习SDN的指路明灯。感谢网络管理中心的其他同窗,舍友亦或同学,大家在网管这个大家庭里相聚相知,谢谢你们的理解和支持。

衷心感谢我的父母和家人,你们是我力量的来源。陪伴你们的时间太少,你们默默奉献无私的爱和帮助,润物细无声,让我无忧无虑地完成研究生学业。感谢我的朋友们,谢谢你们给予我学习上的鼓励和帮助,以及生活上的包容和理解。

感谢相聚的时光,感谢出现在研究生岁月里的人和事。

最后,衷心感谢在百忙之中抽出宝贵时间对论文进行评阅的各位专家和老师!

\end{acknowledgement}
