\chapter{基于SDN的云平台架构设计}
基于SDN的云平台多租户虚拟网络定制化的研究,主要是将SDN集中控制和可编程的优势,集成到云数据中心,通过网络虚拟化技术,为租户提供相互隔离的vSDN网络,vSDN网络由租户自有的控制器实现集中管控。物理网络的集中管控,方便运营商对数据中心网络的管理和维护;租户vSDN网络的集中管控,便于租户利用控制器开发适合自己的上层应用。本章主要对系统架构进行详细介绍。
\section{关键技术分析}
\subsection{需求分析}
基于SDN的云平台多租户虚拟网络定制化的实现,首先解决的是物理网络资源的虚拟化,其次是在虚拟化的基础上进行虚拟网络的带宽时延测量,随后根据测量数据进行链路的定制化,最后为方便用户的操作,完成前端Web界面的设计与实现。系统的整体需求如图\ref{fig:demand}所示。

\begin{figure}[!htb]
  \centering
  \includegraphics[width=0.7\textwidth]{logo/demand}
  \caption{系统需求分析图}
  \label{fig:demand}
\end{figure}

网络虚拟化,需要实现物理资源的隔离,以此同时,还要满足虚拟网络支持SDN模式,本文选用支持资源虚拟化的SDN控制器来完成该功能,通过对底层物理网络下发的流表,添加租户ID等相关信息,完成底层网络中租户隔的离工作。通过自身维护物理网络与虚拟网络的映射表,为上层SDN交换机提供支持OpenFlow协议的虚拟网络,此时租户便可利用自身的SDN控制器完成对vSDN网络的集中管控,进而实现自有的定制化操控。

在虚拟资源之上进行链路带宽时延等信息的测量工作,需要实现SDN模式下链路带宽、时延的精确测量,由于SDN的集中控制和可编程的先天优势,控制器端可以看到全局的网络拓扑,与此同时,OpenFlow协议的规范性和便捷性,方便了我们实现可用带宽、已用带宽、时延的精确测量工作,测量的数据成为用户做出决策的考量值。

链路的定制化,需要用户根据链路的时延、带宽等信息,选取便于传输的最优链路,通过定制化流表的下发,租户便可以对数据中心的虚拟机,自定义数据的传输路径。

前端的Web界面方便了用户的操作,用户可以在web界面,完成对数据中心网络的定制化操作。包括物理、虚拟网络的拓扑获取,链路带宽、时延的前端展示,虚拟网络的创建,定制化链路的前端选取等功能。一系列的操作,需要实现租户之间的隔离,为此我们需要添加认证模块,对租户的请求进行认证解析。实现租户操作的安全性和隔离性。

\subsection{关键技术}
本文旨在构建一个基于SDN的多租户虚拟网络定制化的云平台,该平台一方面实现对数据中心网络的集中控制,方便运营商对数据中心网路的管理和运维,加快新业务的引入速度,运营商可以通过可控的软件部署相关的功能,该自动化部署和运维故障诊断,减少了网络的人工干预,降低了网络的运营费用,也降低了出错率。与此同时,通过虚拟化技术,基于底层的统一物理资源,虚拟出相互隔离的租户vSDN网络,vSDN网络由租户自己的SDN控制器实现集中管控,方便租户的自定义操控,同时,租户可以为自己的控制器开发特有的北向应用,从而定制化数据中心业务需求,当然租户对数据中心虚拟网络的集中控制,便于租户实现对自有网络的监控工作,并可根据当前网络负载定制化链路的转发路径,实现数据包的最优化传输。为实现上述功能,需要突破以下关键技术。

\begin{enumerate}
\item 基于SDN的云平台系统搭建。将SDN集成到云平台数据中心,系统的搭建尤为重要,如何在实现相应功能,满足租户需求的情况下,搭建一套性能稳定的系统成为本文的关键所在。为此,本文创新性地提出了基于SDN的云平台多租户虚拟网络定制化机制,在实现对数据中心网路集中控制的同时,利用虚拟化技术虚拟出相互隔离的vSDN网络,该网路由租户自有的控制器实现集中管控,这样,租户便可以利用自己的SDN控制器完成对自有网络的灵活控制。为方便与OpenStack云平台的集成,本文在对OpenStack云平台改动最小、性能降低最低的情况下,完成了整体系统的搭建,系统的虚拟化平台与OpenStack网络模块Neutron分工合作,Neutron主要负责OpenStack内部网络资源的管理和控制,而本文中支持SDN模式的虚拟化平台主要负责物理服务器之间的SDN网络的管理和控制,两者协同合作,共同完成SDN与OpenStack的集成工作,开放控制器给租户,实现租户对数据中心虚拟网络的灵活管控。
\item 链路带宽、时延的测量。SDN控制器开放给租户后,利用SDN控制器实现对vSDN网络负载的精确测量,是每一个租户希望看到的,测量数据的精确性,直接决定了数据转发策略制定的合理性。因此,全局网络的带宽、时延数据的测量,在本文中起到关键作用。针对带宽、时延,本文创新性地提出了SDN模式下的测量方法。首先针对链路可用带宽,本文采用包对技术,精确地测量了链路的可用带宽,测试证明,该方法在低带宽模式下具有精确的测量结果。针对时延,本文基于三角架构,通过统计数据包的传输时间差,完成了时延的精确测量。带宽和时延数据,成为了租户进行链路定制化的重要依据,租户根据当前链路的负载情况,可以选取一条最优的链路进行数据传输,在提高云平台数据中心带宽利用率的同时,极大地提高了租户网络的数据传输速率。
\end{enumerate}

\section{系统架构设计}
\subsection{系统总体设计图}
系统主要由5个模块组成。分别为底层网络资源和计算资源模块、网络虚拟化模块、控制器模块、通信模块和GUI模块。总体的架构图如图\ref{fig:artic}所示。

\begin{figure}[!htb]
  \centering
  \includegraphics[scale=0.8]{logo/architecture-a}
  \caption{系统整体架构图}
  \label{fig:artic}
\end{figure}

从图中可以看出,架构底层主要包含支持OpenFlow协议的交换机,以及OpenStack内部的虚拟机资源,交换机既包含OpenStack内部的虚拟网桥,亦包含连接物理服务器之间的OpenFlow交换机,将这部分网络开放给租户,主要是为了便于租户实现对数据中心物理网络的可控性。租户可以根据链路时延、带宽定制化最优数据传输链路。网络虚拟化模块,本文基于支持网络虚拟化的SDN控制器实现,一方面实现对底层网络和计算资源的集中控制。另一方面,可以创建相互隔离的租户虚拟网络,该虚拟网络支持SDN模式,租户可利用控制器实现对虚拟网络的集中管控。自主开发北向应用,实现特有的功能。

控制器模块主要实现对租户控制器的管理。租户控制器实现对自有vSDN网络的集中管控,该控制器运行于特定虚拟机之中,租户可以登录虚拟机,进行相应的配置和开发工作。每一个vSDN网络对应唯一的SDN控制器。现有的控制器镜像,已经为其开发了进行链路带宽、时延测量的北向应用。

通信模块主要实现进程间的异步通信,所涉及到的功能有,前端对网络拓扑的获取、带宽时延数据的获取以及虚拟网络创建请求的下发,这一系列的功能均通过通信模块,分别和控制器、网络虚拟化模块完成数据的交互。

前端GUI界面,主要为了方便用户的操作,用户可以在其Web界面,通过最直接方式,完成对数据中心网络的定制化操作。主要包括网络拓扑的显示,链路带宽、时延的前端展示,虚拟网络的创建操作,定制化链路的前端选取等功能。前端模块的实现极大地提高了用户体验效果。

\subsection{系统详细架构图}
在OpenStack云平台数据中心网络中采用SDN架构,核心思想是用控制器对网路运营模式实现统一管控。现有云数据中心网络的模式均为单一节点控制,该模式下的SDN网络对于租户是不可见的。本文提出了多租户虚拟化网络的定制和管理方案,实现了租户自有控制器对虚拟网络的灵活控制。多租户对应多控制器的机制便于租户对自有网络进行带宽时延查询、定制化流表下发、链路切换、控制器北向接口开发实验等自定义操作。系统的详细架构图如图\ref{fig:architecture}所示。

\begin{figure}[!htb]
  \centering
  \includegraphics[width=0.7\textwidth]{logo/architecture}
  \caption{系统详细架构图}
  \label{fig:architecture}
\end{figure}

在兼容现有OpenStack网络模式的前提下,为OpenStack平台的每个计算节点添加br-sdn网桥,由于br-int网桥实际就是一个NORMAL交换机,不管其受SDN控制器控制与否,不会影响其作用,故我们将该网桥以及连接物理服务器之间的支持OpenFlow协议的交换机均受OVX控制器进行管控,OVX控制器实现对网络的集中控制,与此同时,OVX支持SDN模式下的网络虚拟化,可以虚拟出相互隔离的vSDN网络,vSDN网络由租户自己的控制器进行集中控制。前端GUI模块基于OpenStack的Horizon模块进行二次开发,为其添加两个Panel,分别实现对物理网络和虚拟网络的拓扑展示,全局链路带宽、时延展示,以及虚拟网络创建和删除的前端操作。

在该架构图中,OVX与OpenStack中的Neutron组件并列存在,Neutron主要管控OpenStack内部的虚拟网络,包括各种防火墙、Router、DNS服务,而OVX主要负责连接服务器之间的物理网络的管理,两者并列存在。相互协同工作。这种实现模式可以在对OpenStack云平台改动最小、性能影响最低的情况下完成SDN与OpenStack的集成。

对于数据传输,现有架构下,有两种模式可以选择。租户可以选择使用传统网络模式进行数据的传输,数据经br-int和br-tun,由传统模式交换机,进行数据的传输。与此同时,租户亦可通过流表的下发,实现SDN模式下的数据传输,该模式下,数据经由br-int和br-sdn,途径支持OpenFlow协议的交换机,完成数据的转发工作。两种模式的并存,租户可以自主选择切换模式。现有的SDN模式,相比于传统模式,SDN模式下租户可以开发自己的北向应用,实现数据中心网络特定功能的开发。当然SDN现有模式仅仅实现了二层转发,跨路由的传输现阶段只能通过传统模式实现。

\subsection{系统流程图}
系统的整体流程图如图\ref{fig:workflow}所示。流程图主要对系统整体的工作流进行了阐述。详细讲解了从虚拟网络创建到链路定制化的整个流程。

\begin{figure}[!htb]
  \centering
  \includegraphics[width=0.7\textwidth]{logo/workflow}
  \caption{系统流程图}
  \label{fig:workflow}
\end{figure}

\begin{enumerate}
\item 网络虚拟化平台OVX实现对底层物理网络管理,完成物理网络的集中控制。本研究为其开发了创建虚拟网络的API接口\emph{createCustomNetwork},租户可以调用该接口,实现虚拟网络的创建。OVX自身维护虚拟网络与物理网络的映射表,完成南北向指令的映射工作。
\item 从OpenStack Neutron侧,获取虚拟机以及虚拟机绑定br-int网桥的端口信息,OVX本身只能获取到交换机的连接信息,两者结合,得到全网的完整拓扑信息图。基于此物理网络信息,进行虚拟网路的创建。
\item 本研究对OVX封装了创建虚拟网路的\emph{createCustomNetwork}接口,调用该API接口,完成虚拟网络的创建。该虚拟网络基于底层的物理资源,可以是底层物理资源的同构子图,也可以是自定义的虚拟网络拓扑图。
\item 在网络虚拟化平台OVX侧,每个虚拟网络由\emph{virtualnetworkid}进行标识。在OpenStack中,每个租户有自己的ID号\emph{tenantid}。本文将两者的对应关系存储到数据库中。另外,对创建的虚拟网络以及虚拟网络和物理网络的对应关系,亦存储到数据库中,方便用户进行数据的读取和修改。本研究采用MongoDB数据库完成上述操作。由于MongoDB为非关系型数据库,该数据库釆
用BSON语法格式存储数据,可以实现对海量数据的存储管理,它的特点是高性能、易部署和易使用,非常适合存储大数据\cite{mongodb}。

\item OpenStack Neutron侧发送指令到SDN控制器。现阶段,指令主要包括带宽/延迟测量以及定制化流表的下发。未来我们的平台将提供更多的功能。
\item SDN控制器接收到消息,通过我们开发的应用,完成相应的测量工作。最后返回对应的结果。租户可以登录到SDN控制器所在的虚拟机,开发自己的应用程序,以实现更多的功能。
\item GUI界面完成物理网络、虚拟网络、链路带宽、时延等信息的获取,在前端界面完成展示工作。
\end{enumerate}

\section{模块详解}
\subsection{网络虚拟化模块}
网络虚拟化允许租户共用底层的基础设施资源,通过逻辑的隔离,租户只能访问自有的虚拟网络,以确保租户通信的安全性。通过虚拟化技术,可以在单一物理资源之上创建相互隔离的用户组,并且该用户组在网络中保持极高的可扩展性和安全性。网络虚拟化技术可以提高数据中心的运行效率,加速业务的部署。提供更为便捷的业务更新。

本文选用的网络虚拟化平台,通过为每个租户提供一个可访问的虚拟网络拓扑和一个完整的网络头空间来完成虚拟网络的创建,呈现OpenFlow网络给租户,同时经由南向的OpenFlow接口控制底层的物理基础设施,完整的网络头空间保证了租户虚拟网络功能性和流量隔离\cite{Virtual-3}。与传统网络切片进行虚拟化相比,该模式允许不通租户使用相同的网络属性(诸如:IP、TCP/UDP端口等),而且,网络切片所创建的虚拟网络必须是物理网络的同构子图,租户不可以自定义虚拟网络拓扑图。网络虚拟化平台在北向租户看来为支持SDN架构的交换机集合,在南向基础物理设施看来,为SDN控制器。网络虚拟化模块主要由四部分组成,分别为面向网络的南向接口、面向租户的北向接口、全局映射表和API服务器\cite{OVX-1}。具体架构如图\ref{fig:virtual}所示。

\begin{figure}[!htb]
  \centering
  \includegraphics[width=0.7\textwidth]{logo/virtual-detail}
  \caption{网络虚拟化模块架构}
  \label{fig:virtual}
\end{figure}

面向OpenFlow网络的南向接口,连接底层基础设施,管理虚拟平台和基础设施之间的通信信道。面向租户的北向接口,呈现vSDN网络给租户,管理租户SDN控制器和vSDN网络之间的通信信道。全局映射表,用于存储物理网络与虚拟网络拓扑的对应关系,并且分别通过南、北向接口连接这两个网络。API服务器,主要用于监听JsonRPC调用,对用户请求作出相应的处理。

虚拟化平台通过南向OpenFlow信道实现与底层物理网络通信,对于底层网络拓扑,虚拟化平台通过定时发送LLDP报文实现链路的探测,最终获取到整个物理网络的拓扑信息。而对于租户vSDN网络,租户控制器下发的LLDP报文,并未直接发送至底层OpenFlow交换机,而是被虚拟化平台接收,虚拟化平台模拟LLDP的整个过程,将租户vSDN网络的拓扑信息告知租户控制器,从而实现租户控制器对vSDN网络的探测。从南向看来,虚拟化平台被看作是SDN控制器,实现对底层OpenFlow网络的集中控制;从北向租户控制器看来,虚拟化平台为支持OpenFlow协议的交换机集合。在图\ref{fig:virtual}中,北向接口开放给租户,而南向接口通过OpenFlow信道实现对底层网络的管控,两者互不可见,通过映射模块建立起对应关系。北向租户控制器消息的下发和南向物理交换机消息的上传均通过查询映射表修改的方式,完成数据的翻译过程。如图\ref{fig:virtual}右侧所示,北向租户控制器消息通过devirtualize()进行去虚拟化处理,而南向交换机消息通过virtualize()进行虚拟化处理。

虚拟化平台中的租户隔离,作为虚拟化的一个核心需求,本研究中通过在MAC字段中添加额外的信息,使不同租户的流量得到隔离。具体来讲,虚拟化平台在对底层SDN网络下发定制化流表,使得数据包进入边缘交换机时,对数据包的MAC地址做了修改,添加了流ID、租户ID、虚拟链路ID。流ID主要用于查询流入虚拟链路的特定流表;租户ID实现不同租户的隔离;虚拟链路ID用于跟踪这条虚拟链路。当数据包传输至目的边缘交换机时,虚拟化平台下发特定流表,将数据包中的MAC地址字段修改为原始地址,数据包顺利到达目的主机。

综上所述,本文中选用的虚拟化平台主要实现以下四个功能。
\begin{itemize}
\item 创建了一个特定拓扑的独立虚拟网络。
\item 租户使用自己的控制器对虚拟网络进行集中控制。
\item 为每个租户使用整个地址空间,租户虚拟网络可以使用相同的IP地址
\item 可以动态地改变运行中的虚拟网络,并且能够在物理失效的情况下,自动恢复。
\end{itemize}


\subsection{通信模块}
通信模块基于消息队列实现,消息队列是一种应用程序与应用程序之间的通信方式。应用程序通过读写队列中的消息来完成相互的通信。消息传递指的是程序之间通过在消息中发送数据进行通信,而不是通过直接调用彼此来通信。队列作为一个消息代理,为应用程序提供用于存储发送或者接受消息的平台。直到消息从队列中被接收端读取为止。具体的通信模块如图\ref{fig:rabbitmq}所示。

\begin{figure}[!htb]
  \centering
  \includegraphics[width=0.7\textwidth]{logo/rabbitmq}
  \caption{通信模块架构图}
  \label{fig:rabbitmq}
\end{figure}

由图中可以看出,生产者绑定Exchange,Exchange是一个消息交换机,它指定消息按什么规则,路由到哪个队列。具体的路由规则由RoutingKey设定,RoutingKey作为路由关键字,Exchange根据这个关键字进行消息投递。队列用于消息的存储,直到消息被消费者从队列中读取完毕\cite{rabbitmq}。

本文基于消息队列实现通信模块,带来诸多好处。首先消息队列具有可持久性,即使接受方未启动,队列中的消息仍然存在,待接收端启动后,继续从消息队列中读取消息,这种允许延后处理消息的能力极大地提高了通信模块的健壮性。与此同时,消息队列的“只送达一次”和“只处理一次”保证了消息处理的安全性。

\subsection{控制器模块}
控制器作为中间桥梁,连接北向应用和底层的OpenFlow网络设备,一方面,SDN控制器实现对底层网络的集中管控,通过南向接口完成流表的下发。控制器的集中控制可以完成对底层网络的状态监控、转发策略制定以及流量调度等功能;另一方面,控制器开放北向可编程接口给用户,用户可以从实际出发,编写特定的应用程序,自定义网络策略。

本研究中,我们提供了基于Ryu控制器的Linux镜像,租户可以基于此镜像进行控制器的创建,租户只需要将自己创建的虚拟网络指向自己的SDN控制器,即可利用自有的控制器实现对虚拟网络的集中控制。租户可以根据自身需求进行控制器北向应用的相关开发工作。本文主要实现了链路时延、带宽的测量,以及定制化流表的下发实验。

本文选用的Ryu控制器,是由日本最大的电信服务提供商NTT主导幵发的基于Python语言的开源控制器,Ryu控制器为用户提供了丰富的API接口,便于开发人员创建新的应用程序,Ryu架构及主要组件如图\ref{fig:ryu}所示\cite{Ryu-1}。协议支持组件主要提供了对各种网络协议的支持,比如:Netconf,OpenFlow, OF-config等;库文件提供了各种版本的OpenFlow协议标准;内置应用提供了必要的网络功能,基于此,用户可以开发出更多定制化的应用程序;REST组件主要为用户提供丰富的北向API接口。

\begin{figure}[!htb]
  \centering
  \includegraphics[width=0.7\textwidth]{logo/ryu}
  \caption{Ryu架构图}
  \label{fig:ryu}
\end{figure}

\subsection{GUI模块}
为方便用户对虚拟网络的操作,本研究为用户开放了前端GUI接口,用户可以在前端实现物理网络、虚拟网络的拓扑获取,虚拟网络的创建和删除,全局网络带宽、时延的测量,以及数据传输链路的选取等操作。开放的GUI界面,可以在用户不理解底层实现的基础上,进行便捷行的操作。

针对GUI模块,本文基于OpenStack的Horizon模块进行二次开发,为其添加了物理网络和虚拟网络显示的模块,在显示拓扑的基础上,实现虚拟网络创建、删除,以及网络负载查询等功能。

为了租户请求的安全性,我们为其添加了认证机制,在租户发出某一请求时,首先会通过自己的用户名和密码向认证模块申请token,认证成功后,会返回该用户的唯一token标识,token中包含租户的基本信息,以及租户的权限信息,如果认证失败,该次请求结束,并返回错误信息。如果成功,租户会将刚才的请求与现有的token值合并,再一次发送至后端服务器进行请求,服务器端对token验证成功后,执行相应的请求,并将请求结果返回给前端。具体的流程如图\ref{fig:credentials}所示。

\begin{figure}[!htb]
  \centering
  \includegraphics[width=0.7\textwidth]{logo/credentials}
  \caption{请求认证图}
  \label{fig:credentials}
\end{figure}

\section{本章小结}
本章主要对系统的架构做了详细的说明,为后面系统功能的实现做了铺垫。首先论文介绍了需求分析和关键技术,对本文研究内容的需求做了简要阐述,介绍了网络虚拟化、链路负载测量、链路定制化、前端Web界面的简单实现。通过对关键技术的分析说明,阐述了本文系统实现的关键点和难点,对系统搭建、链路负载测量的关键性做了分析。随后介绍了系统的整体架构图和流程图,最后对系统的各模块架构做了简单介绍,主要包括网络虚拟化模块、通信模块、控制器模块、GUI模块。